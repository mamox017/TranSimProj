\hypertarget{index_intro_sec}{}\section{Project Transit\+Sim\+: a Proof-\/of-\/\+Concept Transit System Simulator\+: Iteration 1}\label{index_intro_sec}
{\bfseries Introduction}

This software project is a basic transit simulator of the University of Minnesota -\/ Twin Cities Campus Connector bus routes written in C++. It is composed of four main classes\+: \hyperlink{classPassenger}{Passenger}, \hyperlink{classBus}{Bus}, \hyperlink{classStop}{Stop}, and \hyperlink{classRoute}{Route}. There are three classes that help generate passengers\+: \hyperlink{classPassengerFactory}{Passenger\+Factory}, \hyperlink{classPassengerGenerator}{Passenger\+Generator}, and \hyperlink{classRandomPassengerGenerator}{Random\+Passenger\+Generator}. Two classes also help simulate\+: \hyperlink{classLocalSimulator}{Local\+Simulator} and \hyperlink{classSimulator}{Simulator}.

{\bfseries \hyperlink{classPassenger}{Passenger}}

The \hyperlink{classPassenger}{Passenger} object has a number of traits such as a string name\+\_\+, int destination\+\_\+stop\+\_\+id\+\_\+, int waited\+\_\+at\+\_\+stop\+\_\+, int time\+\_\+on\+\_\+bus\+\_\+, and int id\+\_\+. The name\+\_\+ and destination\+\_\+stop\+\_\+id\+\_\+ of the passenger are randomly generated, whereas both the wait\+\_\+at\+\_\+stop\+\_\+ and time\+\_\+on\+\_\+bus\+\_\+ are both initially 0. When a passenger is put on a bus, the time\+\_\+on\+\_\+bus\+\_\+ is incremented from 0 to 1, indicating that the passenger has gotten onto the bus. The wait\+\_\+at\+\_\+stop\+\_\+ is incremented whenever the passenger is at a stop and Update() is called, time\+\_\+on\+\_\+bus\+\_\+ is also incremented when the passenger is already on a bus. Other functions are various getter and setter methods.

{\bfseries \hyperlink{classBus}{Bus}}

The \hyperlink{classBus}{Bus} object travels on routes by having two members as pointers to an incoming route and an outgoing route. The Move() function moves the bus along by subtracting the bus’s given double speed\+\_\+ member variable from the bus’s double distance\+\_\+remaining\+\_\+ member variable whenever Update() is called. When the bus is moved, if the distance\+\_\+remaining\+\_\+ falls below 0, it determines that it is at a stop and the current\+Stop member variable, which holds a \hyperlink{classStop}{Stop} $\ast$, is changed to the next stop. Therefore, the bus will unload and load passengers accordingly with Unload\+Passengers() and Load\+Passenger(\+Passenger $\ast$). Unload\+Passengers() works by dropping off \hyperlink{classPassenger}{Passenger} objects with matching destination\+\_\+stop\+\_\+id with the id of the \hyperlink{classStop}{Stop} object in the current\+Stop member variable. Load\+Passenger(\+Passenger $\ast$) adds \hyperlink{classPassenger}{Passenger} objects that are at the current\+Stop to the list. If the \hyperlink{classBus}{Bus} determines it is at the end of a route, it will check if it has switched routes to the outgoing\+\_\+route\+\_\+ and will do it if it hasn’t been done. If it has been done, complete will be marked true in Is\+Trip\+Complete(). Other functions are various getter and setter methods.

{\bfseries \hyperlink{classStop}{Stop}}

The \hyperlink{classStop}{Stop} object holds a double distance to the next stop, a \hyperlink{classStop}{Stop} $\ast$ next\+Stop to the following stop, an int id\+\_\+, a \hyperlink{classPassenger}{Passenger} $\ast$ passengers\+\_\+ list, and double longitude \& latitude. When a \hyperlink{classBus}{Bus} object arrives at a stop, \hyperlink{classBus}{Bus} will call Load\+Passengers(\+Bus $\ast$) in the Move() function. This function iterates through the passengers\+\_\+ list who are passengers\+\_\+ at the stop and performs Load\+Passenger(\+Passenger $\ast$) on each one to put them onto the bus. Then it removes the \hyperlink{classPassenger}{Passenger} objects from the \hyperlink{classStop}{Stop} passengers\+\_\+ list. There are other methods also, such as Set\+Next\+Stop(\+Stop $\ast$ next) which is mainly important when it is used to set the next stop of the last stop in incoming\+\_\+route\+\_\+ to the first stop in outgoing\+\_\+route\+\_\+ in Move() in \hyperlink{classBus}{Bus}. Other functions are various getter and setter methods.

{\bfseries \hyperlink{classRoute}{Route}}

The \hyperlink{classRoute}{Route} object is an aggregation of \hyperlink{classStop}{Stop} objects. It is mainly composed of getters and setters for accessibility of \hyperlink{classStop}{Stop} objects for the \hyperlink{classBus}{Bus} objects. The \hyperlink{classPassengerGenerator}{Passenger\+Generator} $\ast$ generator\+\_\+ is used by one of the Generator classes to put \hyperlink{classPassenger}{Passenger} objects at stops on the route. When Update() is called, all of the \hyperlink{classStop}{Stop} objects in the route are iterated through and updated also, also the passengers are generated with Generate\+New\+Passengers(). The Clone() function is used to create a copy of the \hyperlink{classRoute}{Route} object for further usage when dealing with simulations. The Is\+At\+End() function checks the current stop of the route with the last stop, if they match, the route is at the end. Other functions are various getters and setter methods.

{\bfseries \hyperlink{classPassengerFactory}{Passenger\+Factory} \& \hyperlink{classPassengerGenerator}{Passenger\+Generator} \& \hyperlink{classRandomPassengerGenerator}{Random\+Passenger\+Generator}}

These functions drive generation of \hyperlink{classPassenger}{Passenger} objects. \hyperlink{classRandomPassengerGenerator}{Random\+Passenger\+Generator} inherits \hyperlink{classPassengerGenerator}{Passenger\+Generator} and has a function Generate\+Passengers() that overrides and creates the Passengers with various different names generated from \hyperlink{classPassengerFactory}{Passenger\+Factory}.

{\bfseries \hyperlink{classLocalSimulator}{Local\+Simulator} \& \hyperlink{classSimulator}{Simulator}}

These functions drive simulation of the software. They instantiate objects, and run them through trials. \hyperlink{classLocalSimulator}{Local\+Simulator} inherits \hyperlink{classSimulator}{Simulator}.

{\bfseries How it all works together\+:}

Below is an image of a U\+ML Diagram



{\bfseries Thanks for reading!} 